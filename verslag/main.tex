\documentclass{mimosis}
\usepackage{vhistory}

\usepackage{metalogo}

\usepackage[final]{pdfpages}

\usepackage{etoolbox}
\usepackage[utf8]{inputenc}

\usepackage{tocloft}
\usepackage[titletoc]{appendix}

\usepackage{graphicx}
\graphicspath{{Pictures/}}

\usepackage[binary-units=true]{siunitx}
\DeclareSIUnit\px{px}

\setlength{\cftbeforechapskip}{3pt}

\sisetup{%
  detect-all           = true,
  detect-family        = true,
  detect-mode          = true,
  detect-shape         = true,
  detect-weight        = true,
  detect-inline-weight = math,
}

\providecommand{\tightlist}{%
  \setlength{\itemsep}{0pt}\setlength{\parskip}{0pt}}

%%%%%%%%%%%%%%%%%%%%%%%%%%%%%%%%%%%%%%%%%%%%%%%%%%%%%%%%%%%%%%%%%%%%%%%%
% Hyperlinks & bookmarks
%%%%%%%%%%%%%%%%%%%%%%%%%%%%%%%%%%%%%%%%%%%%%%%%%%%%%%%%%%%%%%%%%%%%%%%%

\usepackage[%
  colorlinks = true,
  citecolor  = RoyalBlue,
  linkcolor  = RoyalBlue,
  urlcolor   = RoyalBlue,
  ]{hyperref}

\usepackage{bookmark}


%%%%%%%%%%%%%%%%%%%%%%%%%%%%%%%%%%%%%%%%%%%%%%%%%%%%%%%%%%%%%%%%%%%%%%%%
% Fonts
%%%%%%%%%%%%%%%%%%%%%%%%%%%%%%%%%%%%%%%%%%%%%%%%%%%%%%%%%%%%%%%%%%%%%%%%

\ifxetexorluatex
  \setmainfont{Minion Pro}
\else
  \usepackage[lf]{ebgaramond}
  \usepackage[oldstyle,scale=0.7]{sourcecodepro}
  \singlespacing
\fi

\renewcommand{\th}{\textsuperscript{\textup{th}}\xspace}

%pcb Printed circuit board
%mppt Maximum power point tracking
%spo2 Peripheral oxygen saturation

\newacronym{PCB}{PCB}{Printed circuit board}
\newacronym{MPPT}{MPPT}{Maximum power point tracking}
\newacronym{SPO2}{SPO2}{Peripheral oxygen saturation}
\newacronym{SMD}{SMD}{Surface-mounted Device}
\newacronym{USB}{USB}{Universal Serial Bus}

\makeindex
\makeglossaries

%%%%%%%%%%%%%%%%%%%%%%%%%%%%%%%%%%%%%%%%%%%%%%%%%%%%%%%%%%%%%%%%%%%%%%%%
% Incipit
%%%%%%%%%%%%%%%%%%%%%%%%%%%%%%%%%%%%%%%%%%%%%%%%%%%%%%%%%%%%%%%%%%%%%%%%

\title{Eindverslag \textbf{Stage 3devo}}
\subtitle{Verwarmde behuizing voor het 3D printen van HDPE en PP}
\author{Luca van Straaten (18073611)}

\begin{document}

\frontmatter
\begin{titlepage}
    \vspace*{5cm}
    \makeatletter
    \begin{center}
        \begin{Huge}
            \@title
        \end{Huge}\\[0.1cm]
        %
        \begin{Large}
            \@subtitle
        \end{Large}\\
        %
        \emph{door}\\
        \@author
        %
        \vfill
        Dit document is opgesteld voor de stage bij het 3devo. Luca van
        Straaten is een student Elektrotechniek aan de Haagse Hogeschool te
        delft.\\
        \vspace{.5cm}
        Datum: \today\\
        Versie: V 0.1
    \end{center}
    \makeatother
\end{titlepage}

\newpage


\begin{versionhistory}
  \vhEntry{0.1}{08.09.2021}{Luca}{Document aangemaakt}
\end{versionhistory}

\chapter*{Voorwoord}
\addcontentsline{toc}{chapter}{Voorwoord}
%%%%%%%%%%%%%%%%%%%%%%%%%%%%%%%%%%%%%%%%%%%%%%%%%%%%%%%%%%%%%%%%%%%%%%%%

Dit ontwerpdocument is geschreven als documenatie van het eerste stagetraject
van luca van Straaten\\\\

Utrecht, September 2021\\Luca van Straaten


\chapter{Samenvatting}
\label{Samenvatting}
%%%%%%%%%%%%%%%%%%%%%%%%%%%%%%%%%%%%%%%%%%%%%%%%%%%%%%%%%%%%%%%%%%%%%%%%

\begin{center}
   \begin{minipage}{0.5\textwidth}
      \begin{small}
         Dit is het eindverslag voor de eerste stage van Luca van Straaten. De stage vond plaats van maandag 6 september tot en met vrijdag 12 november 2021 bij 3devo in Utrecht.
      \end{small}
   \end{minipage}
   \vspace{0.5cm}
\end{center}

%%%%%%%%%%%%%%%%%%%%%%%%%%%%%%%%%%%%%%%%%%%%%%%%%%%%%%%%%%%%%%%%%%%%%%%%
% \section{Section}
%%%%%%%%%%%%%%%%%%%%%%%%%%%%%%%%%%%%%%%%%%%%%%%%%%%%%%%%%%%%%%%%%%%%%%%%

%%%%%%%%%%%%%%%%%%%%%%%%%%%%%%%%%%%%%%%%%%%%%%%%%%%%%%%%%%%%%%%%%%%%%%%%
% \subsection{Subsection}
%%%%%%%%%%%%%%%%%%%%%%%%%%%%%%%%%%%%%%%%%%%%%%%%%%%%%%%%%%%%%%%%%%%%%%%%


%\begingroup
%    \let\clearpage\relax
%    \glsaddall
%    \printglossary[type=\acronymtype]
%    \addcontentsline{toc}{chapter}{Acroniemen}
%\endgroup
\newpage
\tableofcontents
\newpage
\listoffigures
\addcontentsline{toc}{chapter}{Lijst van figuren}

\chapter{Inleiding}
\label{inleiding}
%%%%%%%%%%%%%%%%%%%%%%%%%%%%%%%%%%%%%%%%%%%%%%%%%%%%%%%%%%%%%%%%%%%%%%%%

\begin{center}
    \begin{minipage}{0.5\textwidth}
        \begin{small}
            Dit is het eindverslag voor de eerste stage van Luca van Straaten.
            De stage vond plaats van maandag 6 september tot en met vrijdag 12
            november 2021 bij 3devo in Utrecht.
        \end{small}
    \end{minipage}
    \vspace{0.5cm}
\end{center}

\noindent Wat moet de stage zijn en waarom voldoet wat ik heb gedaan

%%%%%%%%%%%%%%%%%%%%%%%%%%%%%%%%%%%%%%%%%%%%%%%%%%%%%%%%%%%%%%%%%%%%%%%%
\section{Wat is 3d printen}
%%%%%%%%%%%%%%%%%%%%%%%%%%%%%%%%%%%%%%%%%%%%%%%%%%%%%%%%%%%%%%%%%%%%%%%%

Waarom heb ik gedaan wat ik heb gedaan

%%%%%%%%%%%%%%%%%%%%%%%%%%%%%%%%%%%%%%%%%%%%%%%%%%%%%%%%%%%%%%%%%%%%%%%%
\section{Bestanden downloaden}
%%%%%%%%%%%%%%%%%%%%%%%%%%%%%%%%%%%%%%%%%%%%%%%%%%%%%%%%%%%%%%%%%%%%%%%%

Dit hele project is gewerkt in en is vast gelegd in git. Maar omdat de inhoud
van het project en verslag intellectueel eigendom is van 3devo, staat dat niet
openbaar op GitHub. Als je echter denkt recht te hebben op het inzien van de
bestanden (elektrische schema's, 3D bestanden, code, etc.) dan kunt u daar
beroep op doen door te mailen naar
\href{mailto:lucavanstraaten@icloud.com}{lucavanstraaten@icloud.com}


\mainmatter

\chapter{Analyse van het probleem}
\label{Analyse_van_het_probleem}
%%%%%%%%%%%%%%%%%%%%%%%%%%%%%%%%%%%%%%%%%%%%%%%%%%%%%%%%%%%%%%%%%%%%%%%%

\begin{center}
   \begin{minipage}{0.5\textwidth}
      \begin{small}
         Wat is het probleem.
      \end{small} 
   \end{minipage}
   \vspace{0.5cm}
\end{center}

%%%%%%%%%%%%%%%%%%%%%%%%%%%%%%%%%%%%%%%%%%%%%%%%%%%%%%%%%%%%%%%%%%%%%%%%
% \section{Section}
%%%%%%%%%%%%%%%%%%%%%%%%%%%%%%%%%%%%%%%%%%%%%%%%%%%%%%%%%%%%%%%%%%%%%%%%

%%%%%%%%%%%%%%%%%%%%%%%%%%%%%%%%%%%%%%%%%%%%%%%%%%%%%%%%%%%%%%%%%%%%%%%%
% \subsection{Subsection}
%%%%%%%%%%%%%%%%%%%%%%%%%%%%%%%%%%%%%%%%%%%%%%%%%%%%%%%%%%%%%%%%%%%%%%%%


\chapter{Eisen van het project}
\label{Eisen_van_het_project}
%%%%%%%%%%%%%%%%%%%%%%%%%%%%%%%%%%%%%%%%%%%%%%%%%%%%%%%%%%%%%%%%%%%%%%%%

%%%%%%%%%%%%%%%%%%%%%%%%%%%%%%%%%%%%%%%%%%%%%%%%%%%%%%%%%%%%%%%%%%%%%%%%
\subsection{randvoorwaarden}
% waar moet je je aan houden (en kan je vanuit het project niet veranderen)?
% Voorbeeld: wet- en regelgeving. De voorschriften van de accountant etc.
%%%%%%%%%%%%%%%%%%%%%%%%%%%%%%%%%%%%%%%%%%%%%%%%%%%%%%%%%%%%%%%%%%%%%%%%

Het frame van een ender-5, de motoren van de ender-5, de behuizing, de
warmte-elementen en de positie daar van waren allemaal al ontworpen voordat aan
het project was begonnen. Dus deze aspecten van het project konden in principe niet
veranderen.

%%%%%%%%%%%%%%%%%%%%%%%%%%%%%%%%%%%%%%%%%%%%%%%%%%%%%%%%%%%%%%%%%%%%%%%%
\subsection{functionele wensen}
% wat moet het resultaat kunnen/doen?
% Voorbeeld: 10.000 liter per uur zuiveren; 2500 broden per dag bakken; 500 aankopen per uur kunnen verwerken etc.
%%%%%%%%%%%%%%%%%%%%%%%%%%%%%%%%%%%%%%%%%%%%%%%%%%%%%%%%%%%%%%%%%%%%%%%%

De verwarmde kamer van de printer mort \SI{160}{\celsius} kunnen woorden.

%%%%%%%%%%%%%%%%%%%%%%%%%%%%%%%%%%%%%%%%%%%%%%%%%%%%%%%%%%%%%%%%%%%%%%%%
\subsection{gebruikerswensen}
% welke eisen stellen gebruikers aan het resultaat?
% Voorbeeld: gebruikers moeten maximaal 3 keer doorklikken op de website om bij de informatie te komen die ze zoeken
%%%%%%%%%%%%%%%%%%%%%%%%%%%%%%%%%%%%%%%%%%%%%%%%%%%%%%%%%%%%%%%%%%%%%%%%

De gebruikersinterface moet een eenvoudige en snelle manier zijn om de printer
te gebruiken, dat betekent in principe dat de printer op dezelfde manier te
bedienen moet zijn als de originele Ender-5

%%%%%%%%%%%%%%%%%%%%%%%%%%%%%%%%%%%%%%%%%%%%%%%%%%%%%%%%%%%%%%%%%%%%%%%%
\subsection{ontwerpbeperkingen}
% eisen die te maken hebben met de bouw/constructie
% Voorbeeld: promotiefilmpje op Youtube mag maximaal 10 minuten lang zijn
%%%%%%%%%%%%%%%%%%%%%%%%%%%%%%%%%%%%%%%%%%%%%%%%%%%%%%%%%%%%%%%%%%%%%%%%

Het project moet woorden gedaan in 10 weken. Dus er kunnen geen onderdelen
woorden gebruikt met een lange levertijd.

De extruder en hotend kunnen niet op de conventionele manier woorden gekoeld
omdat dezen zich in de verwarmde kamer bevinden. Dus als blijkt dat dezen
gekoeld moeten woorden moet daar een andere oplossing voor woorden gevonden.
\chapter{Problemen}
\label{Problemen}
%%%%%%%%%%%%%%%%%%%%%%%%%%%%%%%%%%%%%%%%%%%%%%%%%%%%%%%%%%%%%%%%%%%%%%%%

Tijdens het project zijn eer paar problemen opgetreden. Dezen hadden allemaal
te maken met de extruder en dat daar geen plastic meer uit komt (vast liep).

%%%%%%%%%%%%%%%%%%%%%%%%%%%%%%%%%%%%%%%%%%%%%%%%%%%%%%%%%%%%%%%%%%%%%%%%
\section{Smeltend plastic in de extruder}
\label{s:smeltendplastic}
%%%%%%%%%%%%%%%%%%%%%%%%%%%%%%%%%%%%%%%%%%%%%%%%%%%%%%%%%%%%%%%%%%%%%%%%

Een probleem waar al vrij snel tegen aan werd geloopen is dat de extruder vast
liep door dat het plastic te vroeg smelten, met het gevolg er geen plastic meer uit de nozel kwam.
Dit gebeurde telkens rond de zelfde tijd na het starten van een print.

%%%%%%%%%%%%%%%%%%%%%%%%%%%%%%%%%%%%%%%%%%%%%%%%%%%%%%%%%%%%%%%%%%%%%%%%
\subsection{Oorzaak}
%%%%%%%%%%%%%%%%%%%%%%%%%%%%%%%%%%%%%%%%%%%%%%%%%%%%%%%%%%%%%%%%%%%%%%%%

Dit probleem was het resultaat van warme drive gears (extruder tandwielen).
Heat creep is een term voor het warmte gradiënt door de metalen onderdelen van
de extruder. De stappenmotor van de extruder en de heater cartridge van het
hotend produceren allebei warmte. Deze warmte geleid door de hele extruder
assemblage en komt dus ook bij de drive gears. Als daardoor het plastic ook
warm word smelt het in de extruder. En kan het dus niet meer door de nozel
worden geperst.

%%%%%%%%%%%%%%%%%%%%%%%%%%%%%%%%%%%%%%%%%%%%%%%%%%%%%%%%%%%%%%%%%%%%%%%%
\section{Dubbelvouwend plastic in de extruder}
\label{s:Dubbelvouwend}
%%%%%%%%%%%%%%%%%%%%%%%%%%%%%%%%%%%%%%%%%%%%%%%%%%%%%%%%%%%%%%%%%%%%%%%%

Een probleem tijdens het testen met een PP print wat dat het plastic dubbel
vouwde in de ruimte tussen de extruder tandwielen en de glijder naar de
extruder. Dit is dus een ander probleem dan beschreven in hoofdstuk
\ref{s:smeltendplastic}, dit probleem treed eerder op en zelfs als de kamer
niet is verwarmd. Dit probleem treden ook op tijdens het testen van PP op een
normaale printer, echter om een andere reden.

%%%%%%%%%%%%%%%%%%%%%%%%%%%%%%%%%%%%%%%%%%%%%%%%%%%%%%%%%%%%%%%%%%%%%%%%
\subsection{Oorzaak}
%%%%%%%%%%%%%%%%%%%%%%%%%%%%%%%%%%%%%%%%%%%%%%%%%%%%%%%%%%%%%%%%%%%%%%%%

Dit komt doordat er genoeg ruimte is tussen de extruder tandwielen en het
extruder frame dat er filament tussen door past. En omdat PP veel flexibeler is
dan HDPE, gaat het makkelijk daar tussen zitten.

Dit probleem had een adere oorzaak bij het testen op een standaard ender-3. De
oorzaak was echter dat de extruder veel meer kracht zou moeten zetten omdat de
ender-3 een bowden extruder printer is. Uitleg over wat een bowden tube
extruder is staat in hoofdstuk \ref{ss:Bowden_extruder}.


%%%%%%%%%%%%%%%%%%%%%%%%%%%%%%%%%%%%%%%%%%%%%%%%%%%%%%%%%%%%%%%%%%%%%%%%
\section{Kamer temperatuur overshoot}
%%%%%%%%%%%%%%%%%%%%%%%%%%%%%%%%%%%%%%%%%%%%%%%%%%%%%%%%%%%%%%%%%%%%%%%%

De temperatuur van de verwarmde kamer schiet door het setpoint heen voordat het
stabiliseert of oscilleert rond de ingestelde waarden

%%%%%%%%%%%%%%%%%%%%%%%%%%%%%%%%%%%%%%%%%%%%%%%%%%%%%%%%%%%%%%%%%%%%%%%%
\subsection{Oorzaak}
%%%%%%%%%%%%%%%%%%%%%%%%%%%%%%%%%%%%%%%%%%%%%%%%%%%%%%%%%%%%%%%%%%%%%%%%

De massa van de thermistor is zo hoog dat het lang duurt om in evenwicht te
komen met de luchttemperatuur. Daarom loopt de gemeenten temperatuur dus
aanzienlijk achter op de werkelijke temperatuur. De werkelijke temperatuur is
geverifieerd door een thermokoppel naast de thermistor te hangen.
\chapter{Mogelijke oplossingen}
\label{Mogelijke_oplossingen}
%%%%%%%%%%%%%%%%%%%%%%%%%%%%%%%%%%%%%%%%%%%%%%%%%%%%%%%%%%%%%%%%%%%%%%%%

%%%%%%%%%%%%%%%%%%%%%%%%%%%%%%%%%%%%%%%%%%%%%%%%%%%%%%%%%%%%%%%%%%%%%%%%
\section{Extruder problemen}
%%%%%%%%%%%%%%%%%%%%%%%%%%%%%%%%%%%%%%%%%%%%%%%%%%%%%%%%%%%%%%%%%%%%%%%%

Het probleem van een vastlopende extruder kwam voor met een direct drive
extruder. Een oplossing die overwogen was, was het overstappen naar een andere
extruder architectuur.

%%%%%%%%%%%%%%%%%%%%%%%%%%%%%%%%%%%%%%%%%%%%%%%%%%%%%%%%%%%%%%%%%%%%%%%%
\subsection{Verschillende soorten extruders}
%%%%%%%%%%%%%%%%%%%%%%%%%%%%%%%%%%%%%%%%%%%%%%%%%%%%%%%%%%%%%%%%%%%%%%%%

\begin{figure}[h]
    \centerline{\includegraphics[width=0.85\textwidth]{Basic-diagram-of-FDM-3D-printer-extruder-a-Direct-extruder-b-Bowden-extruder}}
    \caption{Diagram van de twee soorten extruders die veel woorden gebruikt \cite{soorten_extruders}.}
    \label{fig:soorten_extruders}
\end{figure}

%%%%%%%%%%%%%%%%%%%%%%%%%%%%%%%%%%%%%%%%%%%%%%%%%%%%%%%%%%%%%%%%%%%%%%%%
\subsubsection{Bowden extruder}
\label{ss:Bowden_extruder}
%%%%%%%%%%%%%%%%%%%%%%%%%%%%%%%%%%%%%%%%%%%%%%%%%%%%%%%%%%%%%%%%%%%%%%%%

De rechter helft van Figuur \ref{fig:soorten_extruders} \cite{soorten_extruders}
is een diagram van een Bowden extruder. Hierbij is te zien dat \ac{extruder} los
is van de \ac{hotend}.

% uitleg

%%%%%%%%%%%%%%%%%%%%%%%%%%%%%%%%%%%%%%%%%%%%%%%%%%%%%%%%%%%%%%%%%%%%%%%%
\subsubsection{Direct drive extruder}
\label{ss:direct_drive_extruder}
%%%%%%%%%%%%%%%%%%%%%%%%%%%%%%%%%%%%%%%%%%%%%%%%%%%%%%%%%%%%%%%%%%%%%%%%

De linker helft van Figuur \ref{fig:soorten_extruders} \cite{soorten_extruders}
is een diagram van een direct drive extruder.

% uitleg

%%%%%%%%%%%%%%%%%%%%%%%%%%%%%%%%%%%%%%%%%%%%%%%%%%%%%%%%%%%%%%%%%%%%%%%%
\subsection{Zou een Bowden extruder de problemen oplossen}
%%%%%%%%%%%%%%%%%%%%%%%%%%%%%%%%%%%%%%%%%%%%%%%%%%%%%%%%%%%%%%%%%%%%%%%%

Met een bowden extruder zou het probleem van smeltend plastic in de extruder
(Hoofdstuk \ref{s:smeltendplastic}) opgelost kunnen woorden. Echter was
opgemerkt dat PP niet geprint kan woorden met een bowden printer (Hoofdstuk
\ref{s:Dubbelvouwend}).

%%%%%%%%%%%%%%%%%%%%%%%%%%%%%%%%%%%%%%%%%%%%%%%%%%%%%%%%%%%%%%%%%%%%%%%%
\subsection{Zou een water gekoelde extruder de problemen oplossen}
%%%%%%%%%%%%%%%%%%%%%%%%%%%%%%%%%%%%%%%%%%%%%%%%%%%%%%%%%%%%%%%%%%%%%%%%

Een water gekoelde extruder/hotend zou allebei de problemen kunnen oplossen met
als enige nadeel extra complexheid in de vorm van aanstuur elektronica en
software voor de waterkoeling en de waterkoeling zelf (buizen, pomp, radiator).

Een goede optie zou de "Titan Aqua" \cite{titanaqua} zijn.

% bronvermedling naar pagina over watergekoelde hotend/extruder

%%%%%%%%%%%%%%%%%%%%%%%%%%%%%%%%%%%%%%%%%%%%%%%%%%%%%%%%%%%%%%%%%%%%%%%%
\section{Temperatuur overshoot}
%%%%%%%%%%%%%%%%%%%%%%%%%%%%%%%%%%%%%%%%%%%%%%%%%%%%%%%%%%%%%%%%%%%%%%%%

Een \ac{PID} lus voor de kamertemperatuur kan woorden geïmplementeerd in de
firmware van de printer, het vermogen van de warmte-elementen kan worden
teruggedraaid om ervoor te zorgen dat de temperatuur minder snel stijgt en er
dus een minder groot verschil is tussen de gemeenten en de werkelijke
temperatuur.

\chapter{De gekozen oplossing}
\label{De_gekozen_oplossing}
%%%%%%%%%%%%%%%%%%%%%%%%%%%%%%%%%%%%%%%%%%%%%%%%%%%%%%%%%%%%%%%%%%%%%%%%

%%%%%%%%%%%%%%%%%%%%%%%%%%%%%%%%%%%%%%%%%%%%%%%%%%%%%%%%%%%%%%%%%%%%%%%%
\section{Extruder}
%%%%%%%%%%%%%%%%%%%%%%%%%%%%%%%%%%%%%%%%%%%%%%%%%%%%%%%%%%%%%%%%%%%%%%%%

% Om een oplossing to kiezen is een overlecht met het materials team. samen met hen is vervolgen een keuze gemaakt.



%%%%%%%%%%%%%%%%%%%%%%%%%%%%%%%%%%%%%%%%%%%%%%%%%%%%%%%%%%%%%%%%%%%%%%%%
\section{Temperatuur overshoot}
%%%%%%%%%%%%%%%%%%%%%%%%%%%%%%%%%%%%%%%%%%%%%%%%%%%%%%%%%%%%%%%%%%%%%%%%

%%%%%%%%%%%%%%%%%%%%%%%%%%%%%%%%%%%%%%%%%%%%%%%%%%%%%%%%%%%%%%%%%%%%%%%%
% \subsection{Subsection}
%%%%%%%%%%%%%%%%%%%%%%%%%%%%%%%%%%%%%%%%%%%%%%%%%%%%%%%%%%%%%%%%%%%%%%%%


\chapter{Ontwerp van de oplossing en de benodigde onderdelen}
\label{Ontwerp_van_de_oplossing_en_de_benodigde_onderdelen}
%%%%%%%%%%%%%%%%%%%%%%%%%%%%%%%%%%%%%%%%%%%%%%%%%%%%%%%%%%%%%%%%%%%%%%%%

%%%%%%%%%%%%%%%%%%%%%%%%%%%%%%%%%%%%%%%%%%%%%%%%%%%%%%%%%%%%%%%%%%%%%%%%
\section{Hardware}
%%%%%%%%%%%%%%%%%%%%%%%%%%%%%%%%%%%%%%%%%%%%%%%%%%%%%%%%%%%%%%%%%%%%%%%%

het mechanich ontwerp van de printer was grotendeels al gedaan, echter zijn er een aantal aanpassingen gedaan, die staan hier beschreven

%%%%%%%%%%%%%%%%%%%%%%%%%%%%%%%%%%%%%%%%%%%%%%%%%%%%%%%%%%%%%%%%%%%%%%%%
\subsection{Ge-3D-printe oplossingen}
%%%%%%%%%%%%%%%%%%%%%%%%%%%%%%%%%%%%%%%%%%%%%%%%%%%%%%%%%%%%%%%%%%%%%%%%

Een aantal problemen zijn opgelost door kleine onderdelen te 3d printen. hier zijn
daar een paar voorbeelden van.

%%%%%%%%%%%%%%%%%%%%%%%%%%%%%%%%%%%%%%%%%%%%%%%%%%%%%%%%%%%%%%%%%%%%%%%%
\subsubsection{Voetjes van de printer}
%%%%%%%%%%%%%%%%%%%%%%%%%%%%%%%%%%%%%%%%%%%%%%%%%%%%%%%%%%%%%%%%%%%%%%%%

De voetjes van de printer waren te kort, dus daar zijn langere voor
ontworpen en ge-3D-print. Zie Figuur ~\ref{fig:voetjes} voor een render van
het \ac{3d} ontwerp.

\begin{figure}[h]
    \centerline{\includegraphics[width=0.45\textwidth]{voetjes}}
    \caption{Render van het \ac{3d} ontwerp van de voetjes van de printer}
    \label{fig:voetjes}
\end{figure}

%%%%%%%%%%%%%%%%%%%%%%%%%%%%%%%%%%%%%%%%%%%%%%%%%%%%%%%%%%%%%%%%%%%%%%%%
\subsubsection{Afstandhouder}
%%%%%%%%%%%%%%%%%%%%%%%%%%%%%%%%%%%%%%%%%%%%%%%%%%%%%%%%%%%%%%%%%%%%%%%%

De originele printer is omgebouwd met roestvrijstalen panelen aan alle kanten.
Om ervoor te zorgen dat er goede thermische isolatie is van de print kamer, is
het een dubbelwandig ontwerp met glaswol er tussen. De dubbele wanden worden op
afstand gehouden met ge-3D-printe afstandhouders. Zie Figuur
~\ref{fig:afstandhouder} voor een render van het \ac{3d} ontwerp van deze
afstandhouders.

%%%%%%%%%%%%%%%%%%%%%%%%%%%%%%%%%%%%%%%%%%%%%%%%%%%%%%%%%%%%%%%%%%%%%%%%
\subsubsection{Haakje}
%%%%%%%%%%%%%%%%%%%%%%%%%%%%%%%%%%%%%%%%%%%%%%%%%%%%%%%%%%%%%%%%%%%%%%%%

Om de deur dicht te houden is er een haakje geprint. Zie Figuur
~\ref{fig:haakje} voor een render van het \ac{3d} ontwerp van het haakje.

\begin{figure}[h]
    \centering
    \begin{minipage}{0.45\textwidth}
        \centerline{\includegraphics[width=0.9\textwidth]{afstandhouder}}
        \caption{Render van het \ac{3d} ontwerp van de afstandhouder van de printer}
        \label{fig:afstandhouder}
    \end{minipage}\hfill
    \begin{minipage}{0.45\textwidth}
        \centerline{\includegraphics[width=0.9\textwidth]{haakje}}
        \caption{Render van het \ac{3d} ontwerp van het haakje van de printer}
        \label{fig:haakje}
    \end{minipage}
\end{figure}

%%%%%%%%%%%%%%%%%%%%%%%%%%%%%%%%%%%%%%%%%%%%%%%%%%%%%%%%%%%%%%%%%%%%%%%%
\section{Electronica}
%%%%%%%%%%%%%%%%%%%%%%%%%%%%%%%%%%%%%%%%%%%%%%%%%%%%%%%%%%%%%%%%%%%%%%%%

De te gebruiken electronica was grootendeels al vastgesteld voor het project begon.

%%%%%%%%%%%%%%%%%%%%%%%%%%%%%%%%%%%%%%%%%%%%%%%%%%%%%%%%%%%%%%%%%%%%%%%%
% \subsection{Subsection}
%%%%%%%%%%%%%%%%%%%%%%%%%%%%%%%%%%%%%%%%%%%%%%%%%%%%%%%%%%%%%%%%%%%%%%%%


\chapter{Assemblage van de oplossing}
\label{Assemblage_van_de_oplossing}
%%%%%%%%%%%%%%%%%%%%%%%%%%%%%%%%%%%%%%%%%%%%%%%%%%%%%%%%%%%%%%%%%%%%%%%%

%%%%%%%%%%%%%%%%%%%%%%%%%%%%%%%%%%%%%%%%%%%%%%%%%%%%%%%%%%%%%%%%%%%%%%%%
% \section{Section}
%%%%%%%%%%%%%%%%%%%%%%%%%%%%%%%%%%%%%%%%%%%%%%%%%%%%%%%%%%%%%%%%%%%%%%%%

%%%%%%%%%%%%%%%%%%%%%%%%%%%%%%%%%%%%%%%%%%%%%%%%%%%%%%%%%%%%%%%%%%%%%%%%
% \subsection{Subsection}
%%%%%%%%%%%%%%%%%%%%%%%%%%%%%%%%%%%%%%%%%%%%%%%%%%%%%%%%%%%%%%%%%%%%%%%%


\chapter{Problemen waar ik tegen aan ben gelopen}
\label{Problemen_waar_ik_tegen_aan_ben_gelopen}

\section{Hardware}

\subsection{Voetjes van de printer}

De voetjes van de printer waren te kort, dus daar heb ik langere voor ontworpen
en ge 3D print. Zie Figuur ~\ref{fig:voetjes} voor een render van het 3d
ontwerp

\begin{figure}[h]
\centerline{\includegraphics[scale=.5]{voetjes}}
\caption{Render van het 3d ontwerp van de voedjes van de printer}
\label{fig:voetjes}
\end{figure}

\section{Electronica}
\section{Software}

\chapter{Eindresultaat}
\label{Eindresultaat}
%%%%%%%%%%%%%%%%%%%%%%%%%%%%%%%%%%%%%%%%%%%%%%%%%%%%%%%%%%%%%%%%%%%%%%%%

%%%%%%%%%%%%%%%%%%%%%%%%%%%%%%%%%%%%%%%%%%%%%%%%%%%%%%%%%%%%%%%%%%%%%%%%
\section{Section}
%%%%%%%%%%%%%%%%%%%%%%%%%%%%%%%%%%%%%%%%%%%%%%%%%%%%%%%%%%%%%%%%%%%%%%%%

%%%%%%%%%%%%%%%%%%%%%%%%%%%%%%%%%%%%%%%%%%%%%%%%%%%%%%%%%%%%%%%%%%%%%%%%
\subsection{Subsection}
%%%%%%%%%%%%%%%%%%%%%%%%%%%%%%%%%%%%%%%%%%%%%%%%%%%%%%%%%%%%%%%%%%%%%%%%


\chapter{Toetsing eindresultaat aan de hand van de eisen}
\label{Toetsing_eindresultaat_aan_de_hand_van_de_eisen}
%%%%%%%%%%%%%%%%%%%%%%%%%%%%%%%%%%%%%%%%%%%%%%%%%%%%%%%%%%%%%%%%%%%%%%%%

%%%%%%%%%%%%%%%%%%%%%%%%%%%%%%%%%%%%%%%%%%%%%%%%%%%%%%%%%%%%%%%%%%%%%%%%
\section{Section}
%%%%%%%%%%%%%%%%%%%%%%%%%%%%%%%%%%%%%%%%%%%%%%%%%%%%%%%%%%%%%%%%%%%%%%%%

%%%%%%%%%%%%%%%%%%%%%%%%%%%%%%%%%%%%%%%%%%%%%%%%%%%%%%%%%%%%%%%%%%%%%%%%
\subsection{Subsection}
%%%%%%%%%%%%%%%%%%%%%%%%%%%%%%%%%%%%%%%%%%%%%%%%%%%%%%%%%%%%%%%%%%%%%%%%


\chapter{Conclusie en aanbevelingen}
\label{Conclusie_en_aanbevelingen}
%%%%%%%%%%%%%%%%%%%%%%%%%%%%%%%%%%%%%%%%%%%%%%%%%%%%%%%%%%%%%%%%%%%%%%%%

Uit dit verslag kan geconcludeerd worden dat dit project geslaagd is omdat het
product op tijd werd opgeleverd, en voldoet aan alle gestelde eisen.

Echter zijn er wel wat aandachtspunten, zoals dat de printqwalitijd niet voledig
naar wens is, daarom is de aanbeveling aan 3devo om:

\begin{itemize}
    \item Een watergekoelde extruder te installeren
    \item De printer beter te tunen om met gewenste kwaliteit te printen.
\end{itemize}

% This ensures that the subsequent sections are being included as root
% items in the bookmark structure of your PDF reader.
\bookmarksetup{startatroot}

% \bibliographystyle{IEEEtran}
% \bibliography{IEEEabrv,bronnen}

\begin{appendices}
\appendixpage
\noappendicestocpagenum
\addappheadtotoc
% \chapter{Elektrisch schema}
% \includepdf[pages=-,fitpaper,rotateoversize]{Appendices/schema_v2.pdf}
% \chapter{Ontwerp PCB}
% \includepdf[pages=-,fitpaper,rotateoversize]{Appendices/pcb_v2.pdf}
\end{appendices}

\backmatter
\end{document}
