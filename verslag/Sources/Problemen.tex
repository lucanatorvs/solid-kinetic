\chapter{Problemen}
\label{Problemen}
%%%%%%%%%%%%%%%%%%%%%%%%%%%%%%%%%%%%%%%%%%%%%%%%%%%%%%%%%%%%%%%%%%%%%%%%

Tijdens het project zijn een paar problemen opgetreden. Dezen hadden allemaal
te maken met de extruder en dat daar geen plastic meer uit kwam (vast liep).

%%%%%%%%%%%%%%%%%%%%%%%%%%%%%%%%%%%%%%%%%%%%%%%%%%%%%%%%%%%%%%%%%%%%%%%%
\section{Smeltend plastic in de extruder}
\label{s:smeltendplastic}
%%%%%%%%%%%%%%%%%%%%%%%%%%%%%%%%%%%%%%%%%%%%%%%%%%%%%%%%%%%%%%%%%%%%%%%%

Een probleem waar al vrij snel tegen aan werd gelopen is dat de extruder vast
liep door dat het plastic te vroeg smolt, met het gevolg er geen plastic meer uit de nozel kwam.
Dit gebeurde telkens rond de zelfde tijd na het starten van een print.

%%%%%%%%%%%%%%%%%%%%%%%%%%%%%%%%%%%%%%%%%%%%%%%%%%%%%%%%%%%%%%%%%%%%%%%%
\subsection{Oorzaak}
%%%%%%%%%%%%%%%%%%%%%%%%%%%%%%%%%%%%%%%%%%%%%%%%%%%%%%%%%%%%%%%%%%%%%%%%

Dit probleem was het resultaat van warme drive gears (extruder tandwielen).
Heat creep is een term voor het warmte gradiënt door de metalen onderdelen van
de extruder. De stappenmotor van de extruder en de heater cartridge van het
hotend produceren allebei warmte. Deze warmte geleid door de hele extruder
assemblage en komt dus ook bij de drive gears. Als daardoor het plastic ook
warm wordt smelt het in de extruder, en kan het dus niet meer door de nozel
worden geperst.

%%%%%%%%%%%%%%%%%%%%%%%%%%%%%%%%%%%%%%%%%%%%%%%%%%%%%%%%%%%%%%%%%%%%%%%%
\section{Dubbelvouwend plastic in de extruder}
\label{s:Dubbelvouwend}
%%%%%%%%%%%%%%%%%%%%%%%%%%%%%%%%%%%%%%%%%%%%%%%%%%%%%%%%%%%%%%%%%%%%%%%%

Een probleem tijdens het testen met een PP print was dat het plastic dubbel
vouwde in de ruimte tussen de extruder tandwielen en de glijder naar de
extruder. Dit is dus een ander probleem dan beschreven in hoofdstuk
\ref{s:smeltendplastic}, dit probleem treedt eerder op en zelfs als de kamer
niet is verwarmd. Dit probleem treedt ook op tijdens het testen van PP op een
normale printer, echter om een andere reden.

%%%%%%%%%%%%%%%%%%%%%%%%%%%%%%%%%%%%%%%%%%%%%%%%%%%%%%%%%%%%%%%%%%%%%%%%
\subsection{Oorzaak}
%%%%%%%%%%%%%%%%%%%%%%%%%%%%%%%%%%%%%%%%%%%%%%%%%%%%%%%%%%%%%%%%%%%%%%%%

Dit komt doordat er genoeg ruimte is tussen de extruder tandwielen en het
extruder frame zodat er filament tussen door past. En omdat PP veel flexibeler is
dan HDPE, gaat het makkelijk daar tussen zitten.

Dit probleem had een andere oorzaak bij het testen op een standaard Ender-3. De
oorzaak was echter dat de extruder veel meer kracht zou moeten zetten omdat de
Ender-3 een bowden extruder printer is. Uitleg over wat een bowden tube
extruder is staat in hoofdstuk \ref{ss:Bowden_extruder}.


%%%%%%%%%%%%%%%%%%%%%%%%%%%%%%%%%%%%%%%%%%%%%%%%%%%%%%%%%%%%%%%%%%%%%%%%
\section{Kamer temperatuur overshoot}
%%%%%%%%%%%%%%%%%%%%%%%%%%%%%%%%%%%%%%%%%%%%%%%%%%%%%%%%%%%%%%%%%%%%%%%%

De temperatuur van de verwarmde kamer schiet door het setpoint heen voordat het
stabiliseert of oscilleert rond de ingestelde waarden

%%%%%%%%%%%%%%%%%%%%%%%%%%%%%%%%%%%%%%%%%%%%%%%%%%%%%%%%%%%%%%%%%%%%%%%%
\subsection{Oorzaak}
%%%%%%%%%%%%%%%%%%%%%%%%%%%%%%%%%%%%%%%%%%%%%%%%%%%%%%%%%%%%%%%%%%%%%%%%

De massa van de thermistor is zo hoog dat het lang duurt om in evenwicht te
komen met de luchttemperatuur. Daarom loopt de gemeten temperatuur dus
aanzienlijk achter op de werkelijke temperatuur. De werkelijke temperatuur is
geverifieerd door een thermokoppel naast de thermistor te hangen.