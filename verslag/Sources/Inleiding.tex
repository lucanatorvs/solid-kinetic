\chapter{Inleiding}
\label{inleiding}
%%%%%%%%%%%%%%%%%%%%%%%%%%%%%%%%%%%%%%%%%%%%%%%%%%%%%%%%%%%%%%%%%%%%%%%%

\begin{center}
    \begin{minipage}{0.5\textwidth}
        \begin{small}
            Waar de reden tot creatie van dit verslag worden blootgelegd, en daarbij de opdracht en het probleem worden beschreven.
        \end{small}
    \end{minipage}
    \vspace{0.5cm}
\end{center}

\noindent De opdracht is om de aanstuur elecronica te maaken voor een 3D-printer die \ac{hdpe} kan printen.

%%%%%%%%%%%%%%%%%%%%%%%%%%%%%%%%%%%%%%%%%%%%%%%%%%%%%%%%%%%%%%%%%%%%%%%%
\section{Wat is 3D-printen}
%%%%%%%%%%%%%%%%%%%%%%%%%%%%%%%%%%%%%%%%%%%%%%%%%%%%%%%%%%%%%%%%%%%%%%%%

3D-printen is een additieve productie (\ac{am}) methode. Dat betekent dat een product word opgebouwd vanaf nul, in tegenstelling tot subtractieve productie is er weinig of geen verlies van materiaal.
Een andere vorm van additieve productie is bijvoorbeeld spuitgieten, maar hierbij zijn kostbare gietmallen nodig. 3D-printen vereist geen grote investering voor elk nieuw ontwerp, en is daarom uitermate geschikt om snel prototypen te maken. \cite{ATTARAN2017677}

%%%%%%%%%%%%%%%%%%%%%%%%%%%%%%%%%%%%%%%%%%%%%%%%%%%%%%%%%%%%%%%%%%%%%%%%
\subsection{\ac{fdm} 3D-printen}
%%%%%%%%%%%%%%%%%%%%%%%%%%%%%%%%%%%%%%%%%%%%%%%%%%%%%%%%%%%%%%%%%%%%%%%%

Er zijn verschillende vormen van 3D-printen, maar verreweg de meest populaire bij hobbyisten en kleinschalige bedrijven is FDM 3D-printen. FDM staat voor Fused deposition modeling, hierbij word steeds een laag materiaal op de vorige laag neergelegd (depositie) en daar op vast gesmolten (gefuseerd). Door herhaaldelijk laagjes op elkaar neer te leggen kan een \ac{3d} object woorden opgebouwd.

%%%%%%%%%%%%%%%%%%%%%%%%%%%%%%%%%%%%%%%%%%%%%%%%%%%%%%%%%%%%%%%%%%%%%%%%
\subsection{3D-printer software}
%%%%%%%%%%%%%%%%%%%%%%%%%%%%%%%%%%%%%%%%%%%%%%%%%%%%%%%%%%%%%%%%%%%%%%%%



%%%%%%%%%%%%%%%%%%%%%%%%%%%%%%%%%%%%%%%%%%%%%%%%%%%%%%%%%%%%%%%%%%%%%%%%
\subsection{3D-printer firmware}
%%%%%%%%%%%%%%%%%%%%%%%%%%%%%%%%%%%%%%%%%%%%%%%%%%%%%%%%%%%%%%%%%%%%%%%%

%%%%%%%%%%%%%%%%%%%%%%%%%%%%%%%%%%%%%%%%%%%%%%%%%%%%%%%%%%%%%%%%%%%%%%%%
\section{Wat is HDPE}
%%%%%%%%%%%%%%%%%%%%%%%%%%%%%%%%%%%%%%%%%%%%%%%%%%%%%%%%%%%%%%%%%%%%%%%%

%%%%%%%%%%%%%%%%%%%%%%%%%%%%%%%%%%%%%%%%%%%%%%%%%%%%%%%%%%%%%%%%%%%%%%%%
\section{Waarom HDPE}
%%%%%%%%%%%%%%%%%%%%%%%%%%%%%%%%%%%%%%%%%%%%%%%%%%%%%%%%%%%%%%%%%%%%%%%%

nylon, ceramics, wax, bronze, stainless steel, cobalt chrome and titanium.

%%%%%%%%%%%%%%%%%%%%%%%%%%%%%%%%%%%%%%%%%%%%%%%%%%%%%%%%%%%%%%%%%%%%%%%%
\section{Bestanden downloaden}
%%%%%%%%%%%%%%%%%%%%%%%%%%%%%%%%%%%%%%%%%%%%%%%%%%%%%%%%%%%%%%%%%%%%%%%%

Dit hele project is gewerkt in en is vast gelegd in git. Maar omdat de inhoud
van het project en verslag intellectueel eigendom is van 3devo, staat dat niet
openbaar op GitHub. Als je echter denkt recht te hebben op het inzien van de
bestanden (elektrische schema's, 3D bestanden, code, etc.) dan kunt u daar
beroep op doen door te mailen naar
\href{mailto:lucavanstraaten@icloud.com}{lucavanstraaten@icloud.com}
