\chapter{Inleiding}
\label{inleiding}
%%%%%%%%%%%%%%%%%%%%%%%%%%%%%%%%%%%%%%%%%%%%%%%%%%%%%%%%%%%%%%%%%%%%%%%%

\begin{center}
    \begin{minipage}{0.5\textwidth}
        \begin{small}
            Waar de reden tot creatie van dit verslag worden blootgelegd, de
            opdracht en het probleem worden beschreven en achtergrond
            informatie word gegeven.
        \end{small}
    \end{minipage}
    \vspace{0.5cm}
\end{center}

\noindent De opdracht is om aanstuur elektronica te maken voor een 3D-printer
die \ac{hdpe} en \ac{pp} moet kunnen printen. Deze printer is ongebruikelijk
omdat deze een verwarde kamer krijgt die tot \SI{160}{\celsius} moet kunnen
komen. De hypothese is dat een verwarmde werkruimte van een dusdanig hoge
temperatuur, de problemen oplost die worden bevonden met het printen van
\ac{hdpe} en \ac{pp} met een conventionele printer.

%%%%%%%%%%%%%%%%%%%%%%%%%%%%%%%%%%%%%%%%%%%%%%%%%%%%%%%%%%%%%%%%%%%%%%%%
\section{Wat is 3D-printen}
%%%%%%%%%%%%%%%%%%%%%%%%%%%%%%%%%%%%%%%%%%%%%%%%%%%%%%%%%%%%%%%%%%%%%%%%

3D-printen is een additieve productie (\ac{am}) methode. Dat betekent dat een
product word opgebouwd vanaf nul, in tegenstelling tot subtractieve productie
is er weinig of geen verlies van materiaal.  Een andere vorm van additieve
productie is bijvoorbeeld spuitgieten, maar hierbij zijn kostbare gietmallen
nodig. 3D-printen vereist geen grote investering voor elk nieuw ontwerp, en is
daarom uitermate geschikt om snel prototypen te maken. \cite{ATTARAN2017677}

%%%%%%%%%%%%%%%%%%%%%%%%%%%%%%%%%%%%%%%%%%%%%%%%%%%%%%%%%%%%%%%%%%%%%%%%
\subsection{\ac{fdm} 3D-printen}
%%%%%%%%%%%%%%%%%%%%%%%%%%%%%%%%%%%%%%%%%%%%%%%%%%%%%%%%%%%%%%%%%%%%%%%%

Er zijn verschillende vormen van 3D-printen, maar verreweg de meest populaire
bij hobbyisten en kleinschalige bedrijven is FDM 3D-printen. FDM staat voor
Fused deposition modeling, hierbij word steeds een laag materiaal op de vorige
laag neergelegd (depositie) en daar op vast gesmolten (gefuseerd). Door
herhaaldelijk laagjes op elkaar neer te leggen kan een \ac{3d} object woorden
opgebouwd. Het plastik word gesmolten in het hot-end en word door de extruder
uit het hot-end geperst.

%%%%%%%%%%%%%%%%%%%%%%%%%%%%%%%%%%%%%%%%%%%%%%%%%%%%%%%%%%%%%%%%%%%%%%%%
\subsection{3D-printer software}
%%%%%%%%%%%%%%%%%%%%%%%%%%%%%%%%%%%%%%%%%%%%%%%%%%%%%%%%%%%%%%%%%%%%%%%%

Er is verschillende software en firmware nodig om te kunnen 3d printen. Wat
voor soorten software dat is staat hier onder beschreven.

%%%%%%%%%%%%%%%%%%%%%%%%%%%%%%%%%%%%%%%%%%%%%%%%%%%%%%%%%%%%%%%%%%%%%%%%
\subsubsection{\ac{3d} bestanden}
%%%%%%%%%%%%%%%%%%%%%%%%%%%%%%%%%%%%%%%%%%%%%%%%%%%%%%%%%%%%%%%%%%%%%%%%

Software is nodig om een 3d object te ontwerpen, tijdens dit project is
daarvoor \ac{f360} gebruikt. Ook kunnen bepaalde \ac{3d} objecten woorden
gedownload van \href{https://www.thingiverse.com}{Thingiverse}. Dit zijn
meestal \ac{stl} en \ac{3mf} bestanden.

%%%%%%%%%%%%%%%%%%%%%%%%%%%%%%%%%%%%%%%%%%%%%%%%%%%%%%%%%%%%%%%%%%%%%%%%
\subsubsection{Slicer}
%%%%%%%%%%%%%%%%%%%%%%%%%%%%%%%%%%%%%%%%%%%%%%%%%%%%%%%%%%%%%%%%%%%%%%%%

Gedurende dit project is de slicer \emph{Prusaslicer} gebruikt, alle settings
die gebruikt zijn om te slicen zijn op GitHub bijgehouden.  Een slicer is
software die een \ac{3d} bestand als een \ac{stl} of \ac{3mf}, omzet in een
hoop \ac{2d} lagen. Elke laag is gebruikelijk rond de \SI{0.2}{\milli\meter}
hoog. De printer \emph{tekent} deze lagen en stapelt ze op elkaar, zo word een
\ac{3d} object opgebouwd. In de slicer kan je veel dingen instellen om print
kwaliteit of print snelheid te verbeteren. Instellingen zoals snelheid,
temperatuur, flowrate en een hoop meer kun je tweaken voor een beter resultaat,
deze instellingen zijn meestal printer afhankelijk.

%%%%%%%%%%%%%%%%%%%%%%%%%%%%%%%%%%%%%%%%%%%%%%%%%%%%%%%%%%%%%%%%%%%%%%%%
\subsection{3D-printer firmware}
%%%%%%%%%%%%%%%%%%%%%%%%%%%%%%%%%%%%%%%%%%%%%%%%%%%%%%%%%%%%%%%%%%%%%%%%

Firmware is een specifieke klasse van computersoftware die de controle op laag
niveau biedt voor de specifieke hardware van een apparaat.  Voor minder
complexe apparaten (zoals 3D-printers) kan firmware fungeren als het complete
besturingssysteem van het apparaat.

Een veel gebruike firmware voor 3D-printers is Marlin en dat is ook waar voor
is gekozen voor de 3D-printer.  Marlin is een open source firmware voor de
RepRap-familie van 3D-printers. Het is afgeleid van Sprinter en grbl en werd
een op zichzelf staand open sourcenproject op 12 augustus 2011 met de
Github-release. Marlin heeft een licentie onder GPLv3 en is gratis voor alle
toepassingen \cite{Marlin}. 

%%%%%%%%%%%%%%%%%%%%%%%%%%%%%%%%%%%%%%%%%%%%%%%%%%%%%%%%%%%%%%%%%%%%%%%%
\subsubsection{Compiler}
%%%%%%%%%%%%%%%%%%%%%%%%%%%%%%%%%%%%%%%%%%%%%%%%%%%%%%%%%%%%%%%%%%%%%%%%

Een compiler word gebruikt om de firmware van de 3D-printer die in \ac{cpp} is
geschreven, om te zetten naar machine taal. Machine taal kan op de
microprocessor worden uitgevoerd.  Voor het compileren van Marlin is \emph{Auto
Build Marlin} een goede optie \cite{Auto_Build_Marlin}.


%%%%%%%%%%%%%%%%%%%%%%%%%%%%%%%%%%%%%%%%%%%%%%%%%%%%%%%%%%%%%%%%%%%%%%%%
\section{Wat zijn \ac{hdpe} en \ac{pp}}
%%%%%%%%%%%%%%%%%%%%%%%%%%%%%%%%%%%%%%%%%%%%%%%%%%%%%%%%%%%%%%%%%%%%%%%%

\ac{hdpe} is een hoge dichtheid variant van \ac{pe}. \ac{pe} is een
thermoplastisch polymeer dat in een breed scala aan toepassingen wordt
gebruikt. \ac{hdpe} is dus ook een thermoplast dat kan woorden geëxtrudeerd.\\\

\noindent \ac{pp} is tegenwoordig een van de meest gebruikte plastics. Het is
een polymeer dat voornamelijk wordt gebruikt voor verpakkingen. \ac{hdpe} is
een thermoplast dat kan woorden geëxtrudeerd. Het wordt geproduceerd via
ketengroeipolymerisatie uit het monomeer propyleen.

%%%%%%%%%%%%%%%%%%%%%%%%%%%%%%%%%%%%%%%%%%%%%%%%%%%%%%%%%%%%%%%%%%%%%%%%
\subsection{Waarom \ac{hdpe} en \ac{pp}}
%%%%%%%%%%%%%%%%%%%%%%%%%%%%%%%%%%%%%%%%%%%%%%%%%%%%%%%%%%%%%%%%%%%%%%%%

\ac{hdpe} en \ac{pp} hebben een aantal gunstige eigenschappen die er voor
zorgen dat het makkelijk te bewerken is.  Daarom is de meerderheid van de
gebruikte plastics voor verpakkings-materiaal gemaakt van \ac{hdpe} en \ac{pp}.
Omdat een grote hoeveelheid van de afvalstroom daarom ook uit \ac{hdpe} en
\ac{pp} bestaan is het gunstig om dat te kunnen recyclen om mee te 3D-printen.

%%%%%%%%%%%%%%%%%%%%%%%%%%%%%%%%%%%%%%%%%%%%%%%%%%%%%%%%%%%%%%%%%%%%%%%%
\section{Bestanden downloaden}
%%%%%%%%%%%%%%%%%%%%%%%%%%%%%%%%%%%%%%%%%%%%%%%%%%%%%%%%%%%%%%%%%%%%%%%%

Dit hele project is gewerkt in en is vast gelegd in git. Maar omdat de inhoud
van het project en verslag intellectueel eigendom is van 3devo, staat dat niet
openbaar op GitHub. Als je echter denkt recht te hebben op het inzien van de
bestanden (elektrische schema's, 3D bestanden, code, etc.) dan kunt u daar
beroep op doen door te mailen naar
\href{mailto:lucavanstraaten@icloud.com}{lucavanstraaten@icloud.com}
